\section{Installation and run}\label{section:install}

This section explains building and installation of ICSC on any Linux OS. In the following description Ubuntu 22.04 is used, all operations are given for bash terminal.

ICSC is based on Clang/LLVM tool chain. It uses Google Protobuf library and patched SystemC library. Patched SystemC sources are uploaded into ICSC repository.   

There are two ways to install ICSC: 
\begin{itemize}
\item Installation with install.sh script
\item Manual installation
\end{itemize}

\subsection{Prepare to installation}

ICSC can be installed on Linux OS with:
%
\begin{itemize}
\item C++ compiler supports C++20 (for gcc it is version 9.0.0 or later)
\item CMake version 3.12 or later
\item git to clone ICSC repository
\end{itemize} 
%

Initial step before installation is to setup some folder as {\tt\$ICSC\_HOME} and clone ISCS source repository to {\tt\$ICSC\_HOME/icsc}:
%
\begin{lstlisting}[language=bash]
$ export ICSC_HOME=/home/user/my_iscs_folder
$ export CMAKE_CXX_IMPLICIT_INCLUDE_DIRECTORIES=/usr/include/c++/11
$ git clone https://github.com/intel/systemc-compiler $ICSC_HOME/icsc
\end{lstlisting}

{\tt CMAKE\_CXX\_IMPLICIT\_INCLUDE\_DIRECTORIES} is required if there are multiple C++ compilers or ICSC cannot find C++ standard headers.


After clone before installation there is the following folder structure:
%
\begin{lstlisting}
$ICSC_HOME
  * icsc
    * cmake           -- CMake files
    * components      -- assertions, fifo and other library components
    * designs         -- folder for user designs with an design template
    * doc             -- user guide latex and pdf files
    * examples        -- a few illustrative examples
    * sc_elab         -- elaborator sources
    * sc_tool         -- ISCS sources
    * systemc         -- patched SystemC 3.0.0 RC sources
    * tests           -- unit tests
    * .gitmodules     -- not intended to be used here, can be removed
    * CMakeLists.txt  -- Cmake file for ICSC tool
    * LICENSE.txt     -- Apache 2.0 WITH LLVM exceptions license
    * README.md       -- Tool description
    * install.sh      -- Installation script
\end{lstlisting}

\subsection{Installation with install.sh script}

The {\tt install.sh} script contains all the stages of manual installation, that includes generating SystemVerilog code for {\tt examples}.

Open bash terminal and run {\tt icsc/install.sh} from {\tt \$ICSC\_HOME} folder:
%
\begin{lstlisting}[language=bash]
$ cd $ICSC_HOME  
$ icsc/install.sh             # download and install all required components
$ cd $ICSC_HOME
$ source setenv.sh            # setup PATH and LD_LIBRARY_PATH
\end{lstlisting}

Before using the installed tool in a new terminal it needs to run {\tt setenv.sh}:
\begin{lstlisting}[language=bash]
$ export ICSC_HOME=/home/user/my_iscs_folder     
$ cd $ICSC_HOME
$ source setenv.sh            # setup PATH and LD_LIBRARY_PATH 
\end{lstlisting}

\subsection{Manual installation}


\subsubsection{Building and installing Protobuf}

Download Protobuf version 3.6.1 or later from \url{https://github.com/protocolbuffers/protobuf/releases} into {\tt \$ICSC\_HOME} folder.

\begin{lstlisting}[language=bash]
$ cd $ICSC_HOME/protobuf-3.13.0
$ mkdir build && cd build
$ cmake ../cmake/ -DCMAKE_BUILD_TYPE=Release -DCMAKE_INSTALL_PREFIX=$ICSC_HOME -DBUILD_SHARED_LIBS=ON
$ make -j8
$ make install
\end{lstlisting}


\subsubsection{Building and installing LLVM/Clang}

Download LLVM 18.1.8 and Clang 18.1.8 from \url{https://releases.llvm.org/download.html#12.0.1} into {\tt\$ICSC\_HOME```} folder.

\begin{lstlisting}[language=bash]
$ cd $ICSC_HOME/build_deps
$ mv clang-18.1.8.src clang
$ mv cmake-18.1.8.src cmake
$ cd llvm-18.1.8.src
$ mkdir build -p && cd build
$ cmake ../ -DLLVM_ENABLE_ASSERTIONS=ON -DLLVM_TARGETS_TO_BUILD=X86 
            -DLLVM_ENABLE_PROJECTS=clang -DCMAKE_BUILD_TYPE=Release -G "Unix Makefiles" 
            -DCMAKE_INSTALL_PREFIX=$ICSC_HOME -DGCC_INSTALL_PREFIX=$GCC_INSTALL_PREFIX 
            -DCMAKE_CXX_STANDARD=17 -DLLVM_INCLUDE_BENCHMARKS=OFF -DLLVM_INCLUDE_TESTS=OFF
$ make -j8
$ make install
\end{lstlisting}

{\tt LLVM\_ENABLE\_ASSERTIONS} – LLVM assertions, can be used in release as well
{\tt CMAKE\_BUILD\_TYPE} - release or debug, use debug to step into Clang sources (much slower)
{\tt GCC\_INSTALL\_PREFIX} - use the correspondent STD library from used gcc (13.2.0), the same for clang compiler!!!

\subsubsection{Building and installing ICSC}

\begin{lstlisting}[language=bash]
$ cd $ICSC_HOME/icsc
$ mkdir build && cd build
$ cmake ../ -DCMAKE_BUILD_TYPE=Release -DCMAKE_INSTALL_PREFIX=$ICSC_HOME
$ make -j8
$ make install
\end{lstlisting}


\subsection{Folders and files after installation}

After installation there is the following folder structure:
%
\begin{lstlisting}
$ICSC_HOME
  * bin                 -- binary utilities, can be removed
  * build               -- build folder, can be removed
  * designs             -- folder for examples, tests and user designs
  * icsc                -- ICSC sources
    * build             -- build folder, can be removed        
    * cmake             -- CMake files
    * components        -- assertions, fifo and other library components
    * doc               -- user guide latex and pdf files
    * examples          -- a few illustrative examples 
    * sc_elab           -- elaborator sources 
    * sc_tool           -- ISCS sources  
    * systemc           -- patched SystemC 3.0.0 RC sources
    * tests             -- unit tests 
    * .gitmodules       -- not intended to be used here, can be removed
    * CMakeLists.txt    -- Cmake file for ICSC tool
    * LICENSE.txt       -- Apache 2.0 WITH LLVM exceptions license
    * README.md         -- Tool description
    * install.sh        -- Installation script
  * include             -- LLVM/Clang, SystemC and other headers
  * lib                 -- tool compiled libraries
  * libexec             -- can be removed
  * lib64               -- tool compiled libraries
  * share               -- can be removed  
  * CMakeLists.txt      -- CMake file for examples, tests and user designs
  * gdbinit-example.txt -- GDB configuration to be copied into ~/.gdbinit
  * README              -- build and run examples, tests and used designs description
  * setenv.sh           -- set environment script for bash terminal  
\end{lstlisting}


\subsection{Run tool for examples and tests}

There are number of examples in {\tt examples} sub-folder:
%
\begin{itemize}
\item {\tt asserts}    -- immediate and temporal SystemC assertions with SVA generation
\item {\tt counter}    -- simple counter with {\tt SC\_METHOD} and {\tt SC\_CTHREAD} processes
\item {\tt decoder}    -- configurable FIFO example
\item {\tt dvcon20}    -- assertion performance evaluation examples
\item {\tt fsm}        -- finite state machine coding 
\item {\tt intrinsic}  -- Verilog code intrinsic example
\item {\tt int\_error}  -- error reporting example, dangling pointer de-reference inside
\item {\tt latch\_ff}   -- simple latch and flip flop with asynchronous reset
\end{itemize}

There are number of unit tests in {\tt tests} sub-folder:
%
\begin{itemize}
\item {\tt const\_prop} -- constant propagation analysis tests
\item {\tt cthread}     -- general tests in {\tt SC\_CTHREAD}
\item {\tt elab\_only}  -- dynamic elaborator tests
\item {\tt method}      -- general tests in {\tt SC\_METHOD}
\item {\tt mif}         -- modular interface tests
\item {\tt misc}        -- extra tests
\item {\tt record}      -- local and members of struct and class type
\item {\tt state}       -- state tests
\item {\tt uniquify}    -- module uniquification tests
\end{itemize}

\subsubsection{Generate SV code for one specific example or design}
%
\begin{lstlisting}[language=bash]
$ cd $ICSC_HOME
$ source setenv.sh                   # setup PATH and LD_LIBRARY_PATH
$ cd build   
$ cmake ../                          # prepare Makefiles 
$ ctest -R DesignTargetName          # run SV generation for DesignTargetName
\end{lstlisting}

where {\tt DesignTargetName} is a target name in {\tt CMakeLists.txt}.


\subsubsection{Generate SV code for all examples, tests and designs}
%
\begin{lstlisting}[language=bash]
$ cd $ICSC_HOME
$ source setenv.sh                   # setup PATH and LD_LIBRARY_PATH
$ cd build   
$ cmake ../                          # prepare Makefiles 
$ ctest -j8                          # compile and run SV generation
\end{lstlisting}

Generated SystemVerilog files are put into {\tt sv\_out} folders. 
For {\tt counter} example:
% 
\begin{lstlisting}[language=bash]
$ cd icsc/examples/counter           # go to counter example folder 
$ cat sv_out/counter.sv              # see generated SystemVerilog file 
\end{lstlisting}

\subsubsection{Run SystemC simulation}

SystemC simulation for examples and tests can be run with:
\begin{lstlisting}[language=bash]
$ cd $ICSC_HOME
$ mkdir -p build && cd build
$ cmake ../                          # prepare Makefiles 
$ make counter                       # compile SystemC simulation for counter example
$ cd icsc/examples/counter           # go to counter example folder
$ ./counter                          # run SystemC simulation 
\end{lstlisting}

\subsection{Run tool for custom design}

To run ICSC for custom design it needs to create a CMakeList.txt file for the project. SystemVeriog generation is performed with {\tt svc\_target} function call. {\tt svc\_target} is CMake function defined in {\tt \$ICSC\_HOME/lib64/cmake/SVC/svc\_target.cmake}. 

The custom design can be placed into {\tt \$ICSC\_HOME/icsc/designs} folder. 
There is an empty design template {\tt\$ICSC\_HOME/icsc/designs/template}. This design template contains {\tt example.cpp} and {\tt dut.h} files. 
In the design template top module is specified as  variable name {\tt dut\_inst} which is instantiated in module {\tt tb}, so full SystemC name {\tt tb.dut\_inst} is provided.

\begin{lstlisting}[language=make,caption=CMakeList.txt file for design template]
# Design template CMakeList.txt file
project(mydesign)

# All synthesizable source files must be listed here (not in libraries)
add_executable(mydesign example.cpp)

# Test source directory
target_include_directories(mydesign PUBLIC $ENV{ICSC_HOME}/examples/template)

# Add compilation options
# target_compile_definitions(mydesign PUBLIC -DMYOPTION)
# target_compile_options(mydesign PUBLIC -Wall)

# Add optional library, do not add SystemC library (it added by svc_target)
#target_link_libraries(mydesign sometestbenchlibrary)

# svc_target will create @mydesign_sctool executable that runs code generation 
# and @mydesign that runs general SystemC simulation
# ELAB_TOP parameter accepts hierarchical name of DUT  
# (that is SystemC name, returned by sc_object::name() method)
svc_target(mydesign ELAB_TOP tb.dut_inst)
\end{lstlisting}
 
