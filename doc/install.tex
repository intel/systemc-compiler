\section{Installation and run}\label{section:install}

This section explains building and installation of ICSC on any Linux OS. In the following description Ubuntu 20.04 is used, all operations are given for bash terminal.

ICSC is based on Clang/LLVM tool chain. It uses Google Protobuf library and patched SystemC library. Patched SystemC sources are uploaded into ICSC repository.   

There are two ways to install ICSC: 
\begin{itemize}
\item Installation with install.sh script
\item Manual installation
\end{itemize}

\subsection{Prepare to installation}

ICSC can be installed on Linux OS with:
%
\begin{itemize}
\item C++ compiler supports C++17 (for gcc it is version 8.0.0 or later)
\item CMake version 3.12 or later
\item git to clone ICSC repository
\end{itemize} 
%

Initial step before installation is to setup some folder as {\tt\$ICSC\_HOME} and clone ISCS source repository to {\tt\$ICSC\_HOME/icsc}:
%
\begin{lstlisting}[language=bash]
$ export ICSC_HOME=/home/user/my_iscs_folder
$ git clone https://github.com/intel/systemc-compiler $ICSC_HOME/icsc
\end{lstlisting}

After clone before installation there is the following folder structure:
%
\begin{lstlisting}
$ICSC_HOME
  * icsc
    * cmake           -- CMake files
    * components      -- assertions, fifo and other library components
    * designs         -- folder for user designs with an design template
    * doc             -- user guide latex and pdf files
    * examples        -- a few illustrative examples
    * sc_elab         -- elaborator sources
    * sc_tool         -- ISCS sources
    * systemc         -- patched SystemC 2.3.3 sources
    * tests           -- unit tests
    * .gitmodules     -- not intended to be used here, can be removed
    * CMakeLists.txt  -- Cmake file for ICSC tool
    * LICENSE.txt     -- Apache 2.0 WITH LLVM exceptions license
    * README.md       -- Tool description
    * install.sh      -- Installation script
\end{lstlisting}

\subsection{Installation with install.sh script}

The {\tt install.sh} script contains all the stages of manual installation, that includes generating SystemVerilog code for {\tt examples}.

Open bash terminal and run {\tt icsc/install.sh} from {\tt \$ICSC\_HOME} folder:
%
\begin{lstlisting}[language=bash]
$ cd $ICSC_HOME  
$ icsc/install.sh             # download and install all required components
$ cd $ICSC_HOME
$ source setenv.sh            # setup PATH and LD_LIBRARY_PATH
\end{lstlisting}

Before using the installed tool in a new terminal it needs to run {\tt setenv.sh}:
\begin{lstlisting}[language=bash]
$ export ICSC_HOME=/home/user/my_iscs_folder     
$ cd $ICSC_HOME
$ source setenv.sh            # setup PATH and LD_LIBRARY_PATH 
\end{lstlisting}

\subsection{Manual installation}


\subsubsection{Building and installing Protobuf}

Download Protobuf version 3.6.1 or later from \url{https://github.com/protocolbuffers/protobuf/releases} into {\tt \$ICSC\_HOME} folder.

\begin{lstlisting}[language=bash]
$ cd $ICSC_HOME/protobuf-3.13.0
$ mkdir build && cd build
$ cmake ../cmake/ -DCMAKE_BUILD_TYPE=Release -DCMAKE_INSTALL_PREFIX=$ICSC_HOME -DBUILD_SHARED_LIBS=ON
$ make -j8
$ make install
\end{lstlisting}


\subsubsection{Building and installing LLVM/Clang}

Download LLVM 7.0.0 and Clang 7.0.0 from \url{https://releases.llvm.org/download.html#7.0.0} into {\tt \$ICSC\_HOME} folder.

\begin{lstlisting}[language=bash]
$ mv $ICSC_HOME/cfe-7.0.0.src $ICSC_HOME/llvm-7.0.0.src/tools/clang
$ cd $ICSC_HOME/llvm-7.0.0.src
$ mkdir build && cd build
$ cmake ../ -DLLVM_ENABLE_ASSERTIONS=ON -DCMAKE_BUILD_TYPE=Release -DCMAKE_INSTALL_PREFIX=$ICSC_HOME
$ make -j8
$ make install
\end{lstlisting}


\subsubsection{Building and installing ICSC}

\begin{lstlisting}[language=bash]
$ cd $ICSC_HOME/icsc
$ mkdir build && cd build
$ cmake ../ -DCMAKE_BUILD_TYPE=Release -DCMAKE_INSTALL_PREFIX=$ICSC_HOME
$ make -j8
$ make install
\end{lstlisting}


\subsection{Folders and files after installation}

After installation there is the following folder structure:
%
\begin{lstlisting}
$ICSC_HOME
  * bin                 -- binary utilities, can be removed
  build                 -- build folder, can be removed
  * icsc                -- ICSC sources
    * build             -- build folder, can be removed        
    * cmake             -- CMake files
    * components        -- assertions, fifo and other library components
    * designs           -- folder for user designs with an design template
    * doc               -- user guide latex and pdf files
    * examples          -- a few illustrative examples 
    * sc_elab           -- elaborator sources 
    * sc_tool           -- ISCS sources  
    * systemc           -- patched SystemC 2.3.3 sources
    * tests             -- unit tests 
    * .gitmodules       -- not intended to be used here, can be removed
    * CMakeLists.txt    -- Cmake file for ICSC tool
    * LICENSE.txt       -- Apache 2.0 WITH LLVM exceptions license
    * README.md         -- Tool description
    * install.sh        -- Installation script
  * include             -- LLVM/Clang, SystemC and other headers
  * lib                 -- tool compiled libraries
  * libexec             -- can be removed
  * lib64               -- tool compiled libraries
  * share               -- can be removed  
  * CMakeLists.txt      -- CMake file for examples, tests and user designs
  * README              -- build and run examples, tests and used designs description
  * setenv.sh           -- set environment script for bash terminal  
\end{lstlisting}


\subsection{Run tool for examples and tests}

There are number of examples in {\tt examples} sub-folder:
%
\begin{itemize}
\item {\tt asserts}    -- immediate and temporal SystemC assertions with SVA generation
\item {\tt counter}    -- simple counter with {\tt SC\_METHOD} and {\tt SC\_CTHREAD} processes
\item {\tt decoder}    -- configurable FIFO example
\item {\tt dvcon20}    -- assertion performance evaluation examples
\item {\tt fsm}        -- finite state machine coding 
\item {\tt intrinsic}  -- Verilog code intrinsic example
\item {\tt int\_error}  -- error reporting example, dangling pointer de-reference inside
\item {\tt latch\_ff}   -- simple latch and flip flop with asynchronous reset
\end{itemize}

There are number of unit tests in {\tt tests} sub-folder:
%
\begin{itemize}
\item {\tt const\_prop} -- constant propagation analysis tests
\item {\tt cthread}     -- general tests in {\tt SC\_CTHREAD}
\item {\tt elab\_only}  -- dynamic elaborator tests
\item {\tt method}      -- general tests in {\tt SC\_METHOD}
\item {\tt mif}         -- modular interface tests
\item {\tt misc}        -- extra tests
\item {\tt record}      -- local and members of struct and class type
\item {\tt state}       -- state tests
\item {\tt uniquify}    -- module uniquification tests
\end{itemize}

\subsubsection{Generate SV code for one specific example or design}
%
\begin{lstlisting}[language=bash]
$ cd $ICSC_HOME
$ source setenv.sh                   # setup PATH and LD_LIBRARY_PATH
$ cd build   
$ cmake ../                          # prepare Makefiles 
$ ctest -R counter                   # run SV generation for counter example
\end{lstlisting}

\subsubsection{Generate SV code for all examples, tests and designs}
%
\begin{lstlisting}[language=bash]
$ cd $ICSC_HOME
$ source setenv.sh                   # setup PATH and LD_LIBRARY_PATH
$ cd build   
$ cmake ../                          # prepare Makefiles 
$ ctest -j8                          # compile and run SV generation
\end{lstlisting}

Generated SystemVerilog files are put into {\tt sv\_out} folders. 
For {\tt counter} example:
% 
\begin{lstlisting}[language=bash]
$ cd examples/counter                # go to counter example folder 
$ cat sv_out/counter.sv              # see generated SystemVerilog file 
\end{lstlisting}

\subsubsection{Run SystemC simulation}

SystemC simulation for examples and tests can be run with:
\begin{lstlisting}[language=bash]
$ cd $ICSC_HOME
$ mkdir build && cd build
$ cmake ../                          # prepare Makefiles 
$ make counter                       # compile SystemC simulation for counter example
$ cd examples/counter                # go to counter example folder
$ ./counter                          # run SystemC simulation 
\end{lstlisting}

\subsection{Run tool for custom design}

To run ICSC for custom design it needs to create a CMakeList.txt file for the project. SystemVeriog generation is performed with {\tt svc\_target} function call. {\tt svc\_target} is CMake function defined in {\tt \$ICSC\_HOME/lib64/cmake/SVC/svc\_target.cmake}. 

The custom design can be placed into {\tt \$ICSC\_HOME/icsc/designs} folder. 
There is an empty design template {\tt\$ICSC\_HOME/icsc/designs/template}. This design template contains {\tt example.cpp} and {\tt dut.h} files. 
In the design template top module is specified as  variable name {\tt dut\_inst} which is instantiated in module {\tt tb}, so full SystemC name {\tt tb.dut\_inst} is provided.

\begin{lstlisting}[language=make,caption=CMakeList.txt file for design template]
# Design template CMakeList.txt file
project(mydesign)

# All synthesizable source files must be listed here (not in libraries)
add_executable(mydesign example.cpp)

# Test source directory
target_include_directories(mydesign PUBLIC $ENV{ICSC_HOME}/examples/template)

# Add compilation options
# target_compile_definitions(mydesign PUBLIC -DMYOPTION)
# target_compile_options(mydesign PUBLIC -Wall)

# Add optional library, do not add SystemC library (it added by svc_target)
#target_link_libraries(mydesign sometestbenchlibrary)

# svc_target will create @mydesign_sctool executable that runs code generation 
# and @mydesign that runs general SystemC simulation
# ELAB_TOP parameter accepts hierarchical name of DUT  
# (that is SystemC name, returned by sc_object::name() method)
svc_target(mydesign ELAB_TOP tb.dut_inst)
\end{lstlisting}
 

\subsection{Tool options and defines}\label{section:tool_options}

To run ICSC tool for the custom project it needs to create CMakeList.txt file. SystemVeriog code generation is done with {\tt svc\_target} function. {\tt svc\_target} is CMake function defined in {\tt \$ICSC\_HOME/lib64/cmake/SVC/svc\_target.cmake}.

ICSC has several options, which can be specified as {\tt svc\_target} parameters:

\begin{itemize}
\item {\tt ELAB\_TOP} – design top module name, it needs to be specified if top module is instantiated outside of {\tt sc\_main()} or if there are more than one modules in {\tt sc\_main()};
\item {\tt MODULE\_PREFIX} – module prefix string, no prefix if not specified, prefix applied for every module excluding Verilog intrinsic (module with {\tt \_\_SC\_TOOL\_VERILOG\_MOD\_\_}) and memory modules (modules with {\tt \_\_SC\_TOOL\_MEMORY\_NAME\_\_});
\item {\tt REPLACE\_CONST\_VALUE} – replace constant with its evaluated value if possible, by default constant variable is used;
\item {\tt INIT\_LOCAL\_VARS} – initialize non-initialized process local variables with zero to avoid latches, that related to CPP data types only, SC data types always initialized with 0;
\item {\tt NO\_SVA\_GENERATE} – do not generate SVA from immediate and temporal SystemC assertions, normally SVA are generated;
\item {\tt NO\_REMOVE\_EXTRA\_CODE} – do not remove unused variable and unused code in generated SV, normally such code is removed to improve readability.
\end{itemize}

ICSC tool provides {\tt \_\_SC\_TOOL\_\_} define for input SystemC project translation. This define used in temporal assertions and other ICSC library modules to have different behavior for simulation and SV generation. {\tt \_\_SC\_TOOL\_\_} can also be used in project code to hide pieces of code which is not targeted for translation to SystemVerilog.

To completely disable SystemC temporal assertion macro {\tt SCT\_ASSERT\_OFF} can be defined. That allows to hide all assertion specific code to meet SystemC synthesizable standard requirements. {\tt SCT\_ASSERT\_OFF} is required if the SystemC design is passed through a tool which includes its own (not patched) SystemC library.


